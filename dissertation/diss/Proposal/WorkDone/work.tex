\section*{\fontsize{18pt}{1}\selectfont Work to be Done}

The main core component of the project is to be able to apply the KVS concept, where the sequence number and flow Id of a packet are the key and the packet is the value, to implement TCP fast recovery. In order to do that, the following sub-tasks must be done:

\begin{enumerate}
	
	\item First, since the project is done on NetFPGA using P4 programming language, both of which I am not familar with, I first have to study them thoroughly and be proficient working with the platform.
	
	\item I would also need to study and gain a better understanding of KVS applications, TCP congestion control and recovery mechanisms, as well as their use cases (e.g. High Frequency Trading, in-datacenter latency sensitive applications, etc.).
	
	\item The next stage will be to design the architecture for the application. This is a non-trivial process. Generally, it will be to try to map the application to a match-action pipeline.
	
	\item The fourth stage involves implementing the architecture in code. The program, which will be written in P4, will have the basic functionalities such as:
	\begin{itemize}
		\item A basic L3 switch function and TCP decoding (send TCP, read the flow information from the table/register)
		
		\item Matching packets to a "\emph{key}" (its flow Id):
			\begin{itemize}
				\item If the key is of a new packet, store it (SET()).
				\item If the key is of a DUP ACK, read the packet and re-send (GET()).
				\item If the number of DUP ACKs is greater than $N$, do not resend. Instead, forward to host as the standard convention of handling TCP.
				\item Do not resend if the flow Id is not in a "selected" table. In other words, use the KVS only if the flow Id matches a predefined set of flow Ids, or a different predefined rule by the user.
			\end{itemize}		
	
	\end{itemize}
	
	\item \textbf{The simulation stage:} perform functionality test by simulating the program using bmv2 or Xilinx simulators to ensure correctness. 
	
	\item After it runs smoothly in the software, it will then be compiled to the hardware. Ensure that the basic functionalities mentioned above work in hardware.
	
	\item \textbf{The hardware stage:} demonstrate interoperability with a software-based client/application.
	
	Possibly, I will implement a drop $1:N$ packets at the server to force DUP ACKs, and I will also be using a synthetic benchmark such as \emph{multilate} \cite{multilate} -- a memcached benchmark -- or OSNT \cite{osnt} as the applications on top of the network. The criterion to evaluate the performance is the flow completion time. There is no intention to use network simulators such as ns2, omnet++, etc. This is outside the scope of this project.
	
	The aim for this stage is to get a working prototype in hardware that supports a single flow and single packet size, and evaluate on its performance.
	
	\item Once the prototype is up and working, I need to implement further extensions to allows the prototype to support a variety of parameters/conditions (which will be discussed in \textbf{Possible Extensions} section).
		
\end{enumerate}
