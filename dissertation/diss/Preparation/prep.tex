\chapter{Preparation}
\textit{In this chapter, I discuss the Transmission Control Protocol in more dteail. This is followed by a discussion of sftware deliverables, formal requirements, existing libraries and tools levarage, workflow and starting point.
}
\section{Transmission Control Protocol}
\subsection{Test}

\section{Software Used}
	\subsection{Programming Languages}
	I used \textbf{P4} and \textbf{Python} for this project.
	
	In this project, I mainly used the \textbf{P4} programming language \cite{p4.org}. It is a language designed to allow the programming of packet forwarding planes. Besides, unlike general purpose languages such as C or Python, P4 is domain-specific with a number of constructs optimized around network data forwarding, hence is well suited to such a network application.
	
	\textbf{Python}: 
	
	I also made use of the \texttt{make} build automation tool to automate project builds, tests and benchmarks.
	\subsection{P4-NetFPGA Platform}
	a
	\subsection{The NetFPGA SUME board}
	
	\subsection{Development Environment}
	\textbf{Git}: I used \texttt{git} for the 
	
\section{Starting Point}
This project uses the knowledge about TCP introduced in the Part IB \textit{Computer Networking} course. 

During the development of this project, I made use of the materials covered in the following Part II and Part III course:
\begin{itemize}
	\item \textit{Principle of Communications} --- Test
	\item \textit{High Performance Networking} --- ;
	\item \textit{\LaTeX \ and Matlab} --- typesetting the project proposal and dissertation.
\end{itemize}

All code was, using the . Apart from 

In terms of , I had little prior experience with P4 Programming Language and the P4-NetFPGA framework.  I had significant prior experience in Python and Git.

\section{Requirements Analysis}
	Empty for now
