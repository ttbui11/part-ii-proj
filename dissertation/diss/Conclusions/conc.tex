\chapter{Conclusion}
\section{Accomplishments}
Overall, the project achieved its aim of implementing and evaluating a programmable switch that is capable of assisting the fast retransmit process of TCP. The switch functionalities were evaluated in two different environments: a simulation environment and a hardware test. 

With the benefit of hindsight, I would have implemented the architecture prior to starting the project and used it as the starting point. This would have enabled me to focus more on evaluating and give more time to explore useful extensions.

\section{Future Work}

Many promising avenues for further improvement were not explored due to time constraints:

\begin{itemize}
	\item \textbf{Supporting multiple packet sizes}. The current design only supports a single packet size, defined at configuration time. It would definitely be more useful if the design could support multiple packet sizes dynamically, without first specifying them.
	
	\item \textbf{Supporting multiple flows}. It would also be useful to have one programmable switch to serve different applications between the same sender and receiver. The current design only supports one single flow since it only has one cache queue. Adding more cache queues would require a more complicated mechanism to signal a specific cache queue. 
	
	\item \textbf{Dynamic configuration}. The configurations of the flow and the packet size are currently pre-defined and embedded within the P4 code. A more flexible design could allow the user to adaptively installing and removing flows to monitor, as well as to configure the flow. 
	
\end{itemize}

\paragraph{Closing Remarks}\ 

This project has been a fascinating opportunity to explore the field of computer networking, especially high-performance networking, comprehend and appreciate the intricacy of TCP congestion control mechanism and the potential of data plane programmability. The project has achieved most of its goals and attempted to investigate and evaluate a modification to assist TCP fast retransmit algorithm, providing a starting point for future improvements. This project has also contributed to my personal development by improving my software engineering and technical writing skills.