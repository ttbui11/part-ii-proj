\chapter*{Proforma}

{\large
\begin{tabular}{ l@{\hskip 3.25em} p{11cm}}
	Name:               & \bf Thanh Bui \\
	College:            & \bf Downing College \\
	Project Title:      & \bf An accelerated, network-assisted TCP fast retransmit \\
\end{tabular}
}
\\
{\large
\begin{tabular}{ l p{11.5cm}}
	Examination:        & \bf Computer Science Tripos --- Part II, June 2019  \\
	Word Count:         & \bf 11000\footnotemark	 \\
	Project Originator: & Dr Noa Zilberman               \\
	Supervisor:         & Dr Noa Zilberman             		\\ 
\end{tabular}
}
\footnotetext{This word count was computed using \texttt{texcount -sum -inc -utf8 -sub=chapter diss.tex} for chapters 1--5.
}

\section*{Original Aims of the Project}
The aim of this project is to investigate the feasibility and effectiveness of a programmable data plane in application to Transmission Control Protocol (TCP) congestion control. More specifically, I aim to design, implement and evaluate a programmable switch to assist the TCP fast retransmit mechanism. The implementation will be evaluated based on a series of tests, including both software and hardware simulations. A performance evaluation will also be provided.  

\section*{Work Completed}
Almost all my success criteria were met, with the exception of demonstrating the interoperability of my implementation with a software-based client/application. The architecture was designed, then implemented in P4. The implementation was tested via three different simulations: SDNet simulation, SUME simulation and hardware simulation. A performance evaluation of the design was provided. Two of the extensions were also completed. 

\section*{Special Difficulties}
The design stage of the architecture took longer than anticipated due to the limitations of SDNet and the SimpleSumeSwitch architecture of the P4-NetFPGA platform. The current P4-NetFPGA only supports programmable packet processing, i.e. operations on packet headers only, while this project used programmable buffering logic, which would require the ability to buffer packets. Hence, my design cannot be fully expressed in P4 alone. To circumvent this, I had to learn to use Verilog in order to design a different architecture by adding additional HDL modules into the framework that allow packet buffering. This rendered me unable to demonstrate the interoperability of my design with a software-based client/application and meet all my success criteria.


\newpage
\section*{Declaration}

I, Thanh Bui of Downing College, being a candidate for Part II of the Computer Science Tripos, hereby declare that this dissertation and the work described in it are my own work, unaided except as may be specified below, and that the dissertation does not contain material that has already been used to any substantial extent for a comparable purpose.

\begin{minipage}[t]{0.4\textwidth}
	\vspace*{1.5cm}  % leave some space above the horizontal line
	\hrule
	\vspace{1mm} % just a bit more whitespace below the line
	\begin{tabular}[t]{l}
		SIGNED
	\end{tabular}
\end{minipage} 
\hspace{2cm}
\begin{minipage}[t]{0.4\textwidth}
	\vspace*{1.5cm}  % leave some space above the horizontal line
	\hrule
	\vspace{1mm} % just a bit more whitespace below the line
	\begin{tabular}[t]{l}
		DATE
	\end{tabular}
\end{minipage}