\chapter{Implementation}
\textit{I give the overview of the project repository. I move on to explain the  }

\section{Repository Overview}

\dirtree{%
	.1 P4-NetFPGA.
		.2 project.
			.3 simple\_sume\_switch.
				.4 hw.
					.5 hdl.
						.6 nf\_datapath.v*.
				.4 test.
					.5 sim\_switch\_default.
						.6 run.py*.
			.3 src.
				.4 tcp\_retransmit.p4*.
				.4 commands.txt*.
			.3 testdata.
				.4 gen\_testdata.py*.
				.4 digest\_data.py*.
				.4 sss\_sdnet\_tuples.py*.
			.3 templates.
				.4 externs.
					.5 <externs-name>.
						.6 hdl.
							.7 <externs-name>\_template.v*.
		.2 lib.
			.3 hw.
				.4 contrib.
					.5 cores.
						.6 sss\_cache\_queues\_v1\_0\_0*.
				.4 std.
					.5 cores.
						.6 output\_arbiter\_v1\_0\_0*.
}


This project will work mainly with a NetFPGA SUME board \cite{zilberman2014netfpga}, using P4 programming language. I will be using the P4-NetFPGA workflow, which provides infrastructure to compile P4 programs to NetFPGA \cite{fpga}. Apart from that, everything else will be built from scratch.


I have no prior experience with either NetFPGA or P4, but this will be mitigated through self-learning in which I will make use of the online tutorials, Google's resources and the P4 community documentation, as well as the experience of my supervisors. 
\\

\section{Software Implementation}
	\subsection{The Parser}
	\subsection{The SimpleSumeSwitch}
	\subsection{The Deparser}

\section{Hardware Implementation}
	
